
\documentclass[twoside]{EPURapport}
\input{include.tex}

\nolistoftables
\thedocument{Rapport de projet de SE}{D�veloppement d'une application Android pour contr�ler un robot avec le microcontr�leur AndroPOD}{Microcontroleur AndroPOD}

\grade{D�partement Informatique\\ 3\ieme{} ann�e\\ 2011 - 2012}

\authors{%
	\category{�tudiants}{%
		\name{Hao ZHOU} \mail{hao.zhou-2@etu.univ-tours.fr}
		\name{Lei SHANG} \mail{lei.shang@etu.univ-tours.fr}
	}
	\details{DI3 2011 - 2012}
}

\supervisors{%
	\category{Encadrants}{%
		\name{Pascal MAKRIS} \mail{pascal.makris@univ-tours.fr}
	}
	\details{Universit� Fran�ois-Rabelais, Tours}
}

\abstracts{AndroPOD est une carte produite par la soci�t� Microchip. Elle sert d'un pont entre les appareils Android et les �quipements qui ont une interface s�rie. Elle peut se connecter en m�me temps, Android, �quipement s�rie et l'ordinateur de d�bogage. En cons�quence, notre application install�e dans le mobile Android peut communiquer avec une �quipement s�rie (un robot dans notre cas), et on peut en m�me temps la d�boguer avec l'ordinateur connect�. Ce rapport d�taille les trois parties principales de notre projet (syst�me Android, AndroPOD et le robot Lynxmotion) et le processus de la r�alisation.}
{AndroPOD, Android, robot}
{AndroPOD is a serial port developped by Microchip for all the Android-based mobile phones. It's just like a bridge between Android and the other equipments who has a serial port. It can be connected with a mobile Android, a serial equipment (a robot in our case) and a computer for debugging. With that, we can use our application which is intalled in the mobile to communicate with the robot, and at the same time, we can aussi debug the application on the connected computer. This report describe the three important parts of the project (syst�me Android, AndroPOD et le robot Lynxmotion) and the process of our implementation.}
{AndroPOD, Android, robot}


\begin{document}

\chapter{Introduction}
Ce projet est d�velopp� dans le cadre de la formation du syst�me d'exploitation en troisi�me ann�e du d�partement informatique. Le but du projet est de pratiquer les connaissances que nous avons acquises durant les s�ances du syst�me d'exploitation. 


Pour cela, nous sommes demand�s � d�velopper une application sous Android pour contr�ler un petit robot au moyen d'un microcontr�leur nomm� AndroPOD. Donc il nous faut �tudier � la fois :
\bigskip
\begin{itemize}
	\item Le d�veloppement sous le syst�me Android
 	\item L'interface de programmation de la carte AndroPOD
 	\item Le format des commandes pour contr�ler le petit robot
\end{itemize}
\bigskip


Notre travail, ainsi que ce rapport, est organis� selon ces trois parties.


\chapter{Syst�me Android}
\input{Ch1.tex}

\chapter{Carte AndroPOD}
\input{Ch2.tex}

\chapter{Robot Lynxmotion}
\input{Ch3.tex}

\chapter{Mise en oeuvre}
\input{Ch4.tex}

\chapter{Difficult�s rencontr�es}
\input{Ch5.tex}

\chapter{Conclusion}
\input{Ch_conclu.tex}

\annexes

\end{document}

